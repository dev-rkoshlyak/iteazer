\begin{problem}{Расстояние до стены - Hard}{стандартный ввод}{стандартный вывод}{4 секунды}{128 мегабайт}

Сегодня Вам предстоит спасать робота Олли из лабиринта. Лабиринт представляет собой таблицу из \t{n} строк, по \t{m} клеток в каждой. Клетки бывают двух видов: дорога и стена.
Оказывается, дроид панически боится стенок, потому что при контакте с ними он теряет управление. К счастью, нам удалось раздобыть карту лабиринта. Пожалуйста, посчитайте для каждой клетки лабиринта Манхэттенское расстояние до ближайшей стены.



\InputFile
В первой строке входного файла задано два числа: \t{n} и \t{m} - количество строк и столбцов в карте лабиринта. \t{1 <= n,m <= 2000}

Далее следует \t{n} строк по \t{m} символов, описывающих лабиринт. Стена задается символом "*" (звездочка), а дорога - " " (пробел).

Для большего понимания смотрите пример входных / выходных данных.

\OutputFile
Выведите \t{n} строк по \t{m} чисел, где \t{j}-ое число в \t{i}-ой строке равно Манхэттенскому расстоянию от \t{j}-ой клетки в \t{i}-ой строке лабиринта до ближайшей стены.


\Note
Манхэттенское расстояние между клетками \t{(a,b)} и \t{(c,d)} равно \t{|a-c|+|b-d|}

Данная задача отличается от предыдущей только большими ограничениями на размер лабиринта.

\end{problem}

